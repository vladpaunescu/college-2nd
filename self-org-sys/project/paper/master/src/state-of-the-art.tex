% vim: set tw=78 sts=2 sw=2 ts=8 aw et ai:

\subsection{Exact algorithms}

The most direct solution would be to try all permutations (ordered combinations) and see which one is cheapest (using brute force search). The running time for this approach lies within a polynomial factor of O(n!), the factorial of the number of cities, so this solution becomes impractical even for only 20 cities. One of the earliest applications of dynamic programming is the Held–Karp algorithm that solves the problem in time $O(n^2 2^n)$.


Solution to a symmetric TSP with 7 cities using brute force search. Note: Number of permutations: (7-1)!/2 = 360
Improving these time bounds seems to be difficult. For example, it has not been determined whether an exact algorithm for TSP that runs in time $O(1.9999^n)$ exists.

Other approaches include:

\begin{itemize}

\item Various branch-and-bound algorithms, which can be used to process TSPs containing 40–60 cities.

\item Progressive improvement algorithms which use techniques reminiscent of linear programming. Works well for up to 200 cities.

\item 

Implementations of branch-and-bound and problem-specific cut generation (branch-and-cut[21]); this is the method of choice for solving large instances. This approach holds the current record, solving an instance with 85,900 cities, see Applegate et al. (2006).

\end{itemize}


An exact solution for 15,112 German towns from TSPLIB was found in 2001 using the cutting-plane method proposed by George Dantzig, Ray Fulkerson, and Selmer M. Johnson in 1954, based on linear programming. The computations were performed on a network of 110 processors located at Rice University and Princeton University (see the Princeton external link). The total computation time was equivalent to 22.6 years on a single 500 MHz Alpha processor. In May 2004, the travelling salesman problem of visiting all 24,978 towns in Sweden was solved: a tour of length approximately 72,500 kilometers was found and it was proven that no shorter tour exists.

In March 2005, the travelling salesman problem of visiting all 33,810 points in a circuit board was solved using Concorde TSP Solver: a tour of length 66,048,945 units was found and it was proven that no shorter tour exists. The computation took approximately 15.7 CPU-years (Cook et al. 2006). In April 2006 an instance with 85,900 points was solved using Concorde TSP Solver, taking over 136 CPU-years, see Applegate et al. (2006).

\subsection{Concorde TSP Solver}
24 years ago a 2392 city example of the TSP was solved in a 23 hour run on a super computer to set a new world record. This same problem now solves in 7 minutes on an iPhone 4!

The Travelling Salesman Problem (TSP) is quite literally a million dollar question. The basic formulation of this math problem is to calculate the shortest possible route to any number of cities and return to the point of origin. It may be surprising to learn that this apparently simple problem has remained unsolved for centuries. With the development of computing in the 20th century, it seemed as though the solution to the TSP would now be found using
computer programming, but the actuality is that this problem is simply too large for any single computer to run through all possible iterations of a tour. In 2006, a real breakthrough came when mathematicians found the optimal solution for an 85,900-city challenge—but this solution relied on the equivalent of 136 years of computer time and cannot accommodate the inclusion of even a single additional city. Now with the development of CONCORDE TSP SOLVER, mathematicians and students have, at their fingertips, a program capable of solving the TSP for a large range of cities in minutes.

\subsection{Heuristic and approximation algorithms}

Various heuristics and approximation algorithms, which quickly yield good solutions have been devised. Modern methods can find solutions for extremely large problems (millions of cities) within a reasonable time which are with a high probability just 2–3\% awaw from the exact solution.

Several categories of heuristics are recognized.

\subsection{Genetic Algorithm}

A genetic algorithm can be used to find a solution is much less time. Although it might not find the best solution, it can find a near perfect solution for a 100 city tour in less than a minute. There are a couple of basic steps to solving the traveling salesman problem using a GA.

First, create a group of many random tours in what is called a population. This algorithm uses a greedy initial population that gives preference to linking cities that are close to each other. 

Second, pick k of the better (shorter) tours parents in the population and combine them to make k new child tours. Hopefully, these children tour will be better than either parent.

A small percentage of the time, the child tours are mutated. This is done to prevent all tours in the population from looking identical. The new child tours are inserted into the population replacing two of the longer tours. The size of the population remains the same. 
New children tours are repeatedly created until the desired goal is reached.
As the name implies, Genetic Algorithms mimic nature and evolution using the principles of Survival of the Fittest. 

\subsection{Ant Colony Optimization}

Artificial intelligence researcher Marco Dorigo described in 1997 a method of heuristically generating "good solutions" to the TSP using a simulation of an ant colony called ACS (Ant Colony System). It models behavior observed in real ants to find short paths between food sources and their nest, an emergent behaviour resulting from each ant's preference to follow trail pheromones deposited by other ants.

ACS sends out a large number of virtual ant agents to explore many possible routes on the map. Each ant probabilistically chooses the next city to visit based on a heuristic combining the distance to the city and the amount of virtual pheromone deposited on the edge to the city. The ants explore, depositing pheromone on each edge that they cross, until they have all completed a tour. At this point the ant which completed the shortest tour deposits virtual pheromone along its complete tour route (global trail updating). The amount of pheromone deposited is inversely proportional to the tour length: the shorter the tour, the more it deposits.

\subsection{Applications}

\subsubsection{Drilling of printed circuit boards}

A direct application of the TSP is in the drilling problem of printed circuit boards (PCBs) (Grötschel et al., 1991). To connect a conductor on one layer with a conductor on another layer, or to position the pins of integrated circuits, holes have to be drilled through the board. The holes may be of different sizes. To drill two holes of different diameters consecutively, the head of the machine has to move to a tool box and change the drilling equipment. This is quite time consuming. Thus it is clear that one has to choose some diameter, drill all holes of the same diameter, change the drill, drill the holes of the next diameter, etc. Thus, this drilling problem can be viewed as a series of TSPs, one for each hole diameter, where the 'cities' are the initial position and the set of all holes that can be drilled with one and the same drill. The 'distance' between two cities is given by the time it takes to move the drilling head from one position to the other. The aim is to minimize the travel time for the machine head.

\subsubsection{Overhauling gas turbine engines}
(Plante et al., 1987) reported this application and it occurs when gas turbine engines of aircraft have to be overhauled. To guarantee a uniform gas flow through the turbines there are nozzle-guide vane assemblies located at each turbine stage. Such an assembly basically consists of a number of nozzle guide vanes affixed about its circumference. All these vanes have individual characteristics and the correct placement of the vanes can result in substantial benefits (reducing vibration, increasing uniformity of flow, reducing fuel 

The problem of placing the vanes in the best possible way can be modeled as
a TSP with a special objective function. 

\subsubsection{X-Ray crystallography}
Analysis of the structure of crystals (Bland \& Shallcross, 1989; Dreissig \& Uebach, 1990) is an important application of the TSP. Here an X-ray diffractometer is used to obtain information about the structure of crystalline material. To this end a detector measures the intensity of X ray
reflections of the crystal from various positions. Whereas the measurement itself can be accomplished quite fast, there is a considerable overhead in positioning time since up to hundreds of thousands positions have to be realized for some experiments. In the two examples that we refer to, the positioning involves moving four motors. The time needed to move from one position to the other can be computed very accurately. The result of the
experiment does not depend on the sequence in which the measurements at the various positions are taken. However, the total time needed for the experiment depends on the sequence. Therefore, the problem consists of finding a sequence that minimizes the total positioning time. This leads to a traveling salesman problem. 

\subsubsection{Computer wiring}
(Lenstra \& Rinnooy Kan, 1974) reported a special case of connecting components on a computer board. Modules are located on a computer board and a given subset of pins has to be connected. In contrast to the usual case where a Steiner tree connection is desired, here the requirement is that no more than two wires are attached to each pin. Hence we have the problem of finding a shortest Hamiltonian path with unspecified starting and terminating points. A similar situation occurs for the so-called testbus wiring. To test the manufactured board one has to realize a connection which enters the board at some specified point, runs through all the modules, and terminates at some specified point. For each module we also have a specified entering and leaving point for this test wiring. This problem also amounts to solving a Hamiltonian path problem with the difference that the distances are not symmetric and that starting and terminating point are specified.

\subsubsection{The order-picking problem in warehouses}
This problem is associated with material handling in a warehouse (Ratliff \& Rosenthal, 1983). Assume that at a warehouse an order arrives for a certain subset of the items stored in the warehouse. Some vehicle has to collect all items of this order to ship them to the customer. The relation to the TSP is immediately seen. The storage locations of the items correspond to the nodes of the graph. The distance between two nodes is given by the time needed to move the vehicle from one location to the other. The problem of finding a shortest route for the vehicle with minimum pickup time can now be solved as a TSP. In special cases this problem can be solved easily, see (van Dal, 1992) for an extensive discussion and for references.

\subsubsection{Vehicle routing}

Suppose that in a city n mail boxes have to be emptied every day within a certain period of time, say 1 hour. The problem is to find the minimum number of trucks to do this and the shortest time to do the collections using this number of trucks. As another example, suppose that n customers require certain amounts of some commodities and a supplier has to satisfy all demands with a fleet of trucks. The problem is to find an assignment of customers to the
trucks and a delivery schedule for each truck so that the capacity of each truck is not exceeded and the total travel distance is minimized. Several variations of these two problems, where time and capacity constraints are combined, are common in many realworld applications. This problem is solvable as a TSP if there are no time and capacity constraints and if the number of trucks is fixed (say m ). In this case we obtain an m - salesmen problem. Nevertheless, one may apply methods for the TSP to find good feasible
solutions for this problem (see Lenstra \& Rinnooy Kan, 1974).

\subsubsection{Mask plotting in PCB production}

For the production of each layer of a printed circuit board, as well as for layers of integrated semiconductor devices, a photographic mask has to be produced. In our case for printed circuit boards this is done by a mechanical plotting device. The plotter moves a lens over a photosensitive coated glass plate. The shutter may be opened or closed to expose specific parts of the plate. There are different apertures available to be able to generate different structures on the board. Two types of structures have to be considered. A line is exposed on the plate by moving the closed shutter to one endpoint of the line, then opening the shutter and moving it to the other endpoint of the line. Then the shutter is closed. A point type structure is generated by moving (with the appropriate aperture) to the position of that
point then opening the shutter just to make a short flash, and then closing it again. Exact modeling of the plotter control problem leads to a problem more complicated than the TSP and also more complicated than the rural postman problem. A real-world application in the actual production environment is reported in (Grötschel et al., 1991).
